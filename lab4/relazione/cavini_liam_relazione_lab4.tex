\documentclass{article}
\usepackage{parskip}
\usepackage{caption}
\usepackage{subcaption}
\usepackage{multirow}
\usepackage{float}

% Language setting
% Replace 'english' with e.g. 'spanish' to change the document language
\usepackage[italian]{babel}
\addto\captionsenglish{\renewcommand{\figurename}{Figura}}

% Set page size and margins
% Replace 'letterpaper' with 'a4paper' for UK/EU standard size
\usepackage[letterpaper,top=2cm,bottom=2cm,left=2cm,right=2cm,marginparwidth=1.75cm]{geometry}

% Useful packages
\usepackage{amsmath}
\usepackage{graphicx}
\graphicspath{ {./images/} }
\usepackage[colorlinks=true, allcolors=blue]{hyperref}


\usepackage{listings}
\usepackage{xcolor}

\definecolor{codegreen}{rgb}{0,0.6,0}
\definecolor{codegray}{rgb}{0.5,0.5,0.5}
\definecolor{codepurple}{rgb}{0.58,0,0.82}
\definecolor{backcolour}{rgb}{0.95,0.95,0.92}

\lstdefinestyle{mystyle}{
    backgroundcolor=\color{backcolour},   
    commentstyle=\color{codegreen},
    keywordstyle=\color{magenta},
    numberstyle=\tiny\color{codegray},
    stringstyle=\color{codepurple},
    basicstyle=\ttfamily\footnotesize,
    breakatwhitespace=false,         
    breaklines=true,                 
    captionpos=b,                    
    keepspaces=true,                 
    numbers=left,                    
    numbersep=5pt,                  
    showspaces=false,                
    showstringspaces=false,
    showtabs=false,                  
    tabsize=2
}

\lstset{style=mystyle}

\begin{document}
\begin{center}
    {\Large Alma Mater Studiorum - Università di Bologna}
    
    \vspace{0.5cm}
    {\bf \large Relazione per il corso di Data Science}
\end{center} 

\noindent
{ Liam Cavini} \hfill {\bf 4° Foglio, Regressione Logistica e Modelli}\\
{\ Semestre Invernale 2024/2025} \hfill 30/10/2024

\subsection*{Risorse}
Il codice utilizzato, insieme al file .tex di questo documento, possono essere trovati nella seguente repository github: \url{https://github.com/LazyLagrangian/data_science}.

\subsection*{Esercizio 1 - Fasi del Modello di Ising con Regressione Logistica}
L'esercizio ha lo scopo di allenare un regressore logistico su un dataset di modelli di Ising 2-D, classificati in ordinati e disordinati.

I dati forniti, che risultano essere ordinati da sistemi con temperature più basse a quelli a temperature più alte (e quindi da più ordinati a più disordinati) sono stati divisi in tre batch.
I primi $70000$ hanno formato il batch contenente i sistemi ordinati, i successivi $30000$ il batch dei sistemi critici (al limite tra ordinati e disordinati), e i dati rimanenti il batch dei sistemi disordinati.

La divisione in batch non è da confondere con la classificazione dei sistemi:
infatti, nonostante ciascun membro del batch ordinato è classificato come ordinato (e analagomente per il batch disordinato), 
non vi è una classificazione a parte per i dati critici, che invece risultano essere o ordinati o disordinati.

\begin{figure}[H]
    \centering
    \begin{subfigure}{.30\textwidth}
       \centering
       \includegraphics[width=1 \textwidth]{immagini/ex_ord.png}
    \end{subfigure}
    \begin{subfigure}{.30\textwidth}
       \centering
       \includegraphics[width=1\textwidth]{immagini/ex_crit.png}
   \end{subfigure} 
   \begin{subfigure}{.30\textwidth}
    \centering
    \includegraphics[width=1\textwidth]{immagini/ex_dis.png}
    \end{subfigure} 
   \caption{\emph{Le tre immagini mostrano sul piano x-y i valori dell}}
   \label{fig:esempio_batch}
\end{figure}

\end{document}