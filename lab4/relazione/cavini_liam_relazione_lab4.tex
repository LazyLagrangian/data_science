\documentclass{article}
\usepackage{parskip}
\usepackage{caption}
\usepackage{subcaption}
\usepackage{multirow}
\usepackage{float}

% Language setting
% Replace 'english' with e.g. 'spanish' to change the document language
\usepackage[italian]{babel}
\addto\captionsenglish{\renewcommand{\figurename}{Figura}}

% Set page size and margins
% Replace 'letterpaper' with 'a4paper' for UK/EU standard size
\usepackage[letterpaper,top=2cm,bottom=2cm,left=2cm,right=2cm,marginparwidth=1.75cm]{geometry}

% Useful packages
\usepackage{amsmath}
\usepackage{graphicx}
\graphicspath{ {./images/} }
\usepackage[colorlinks=true, allcolors=blue]{hyperref}


\usepackage{listings}
\usepackage{xcolor}

\definecolor{codegreen}{rgb}{0,0.6,0}
\definecolor{codegray}{rgb}{0.5,0.5,0.5}
\definecolor{codepurple}{rgb}{0.58,0,0.82}
\definecolor{backcolour}{rgb}{0.95,0.95,0.92}

\lstdefinestyle{mystyle}{
    backgroundcolor=\color{backcolour},   
    commentstyle=\color{codegreen},
    keywordstyle=\color{magenta},
    numberstyle=\tiny\color{codegray},
    stringstyle=\color{codepurple},
    basicstyle=\ttfamily\footnotesize,
    breakatwhitespace=false,         
    breaklines=true,                 
    captionpos=b,                    
    keepspaces=true,                 
    numbers=left,                    
    numbersep=5pt,                  
    showspaces=false,                
    showstringspaces=false,
    showtabs=false,                  
    tabsize=2
}

\lstset{style=mystyle}

\begin{document}
\begin{center}
    {\Large Alma Mater Studiorum - Università di Bologna}
    
    \vspace{0.5cm}
    {\bf \large Relazione per il corso di Data Science}
\end{center} 

\noindent
{ Liam Cavini} \hfill {\bf 4° Foglio, Regressione Logistica e Modelli}\\
{\ Semestre Invernale 2024/2025} \hfill 12/11/2024

\subsection*{Risorse}
Il codice utilizzato, insieme al file .tex di questo documento, possono essere trovati nella seguente repository github: \url{https://github.com/LazyLagrangian/data_science}.

\subsection*{Esercizio 1 - Fasi del Modello di Ising con Regressione Logistica}
L'obiettivo dell'esercizio è quello di allenare un regressore logistico su un dataset di modelli di Ising 2-D, classificati in ordinati e disordinati.

I dati forniti consistono in 160000 sistemi composti da $40\times40$ valori, disposti dalle temperature più basse alle temperature più alte (e quindi da più ordinati a più disordinati). Questi sono stati divisi in tre batch:
i primi $70000$ hanno formato il batch contenente i sistemi ordinati, i successivi $30000$ il batch dei sistemi critici (al limite tra ordinati e disordinati), e i dati rimanenti il batch dei sistemi disordinati.

La divisione in batch non è da confondere con la classificazione dei sistemi:
infatti, nonostante ciascun membro del batch ordinato sia classificato come ordinato (e analogamente per il batch disordinato), 
non vi è una classificazione a parte per i dati critici, che invece risultano essere ordinati o disordinati.

La figura \ref{fig:esempio_batch} mostra sistemi tipici appartenenti a ciascun batch,
la figura \ref{fig:istogrammi} fornisce una panoramica generale del livello di ordine relativo a ciascuno di essi.



\begin{figure}[H]
    \centering
    \begin{subfigure}{.29\textwidth}
       \centering
       \includegraphics[width=1 \textwidth]{immagini/ex_ord.png}
    \end{subfigure}
    \begin{subfigure}{.29\textwidth}
       \centering
       \includegraphics[width=1\textwidth]{immagini/ex_crit.png}
   \end{subfigure} 
   \begin{subfigure}{.29\textwidth}
    \centering
    \includegraphics[width=1\textwidth]{immagini/ex_dis.png}
    \end{subfigure} 
   \caption{\emph{Tre esempi rappresentativi dei batch, ordinato, critico e disordinato. Ogni cella del piano x-y rappresenta uno dei $40\times40$ valori che caratterizzano un sistema.}}
   \label{fig:esempio_batch}
\end{figure}


Per il training sono stati utilizzati solamente i dati appartenenti al batch ordinato e disordinato, che sono stati ulteriormente suddivisi in batch di training e testing.

Si è poi allenato il regressore logistico, utilizzando tre metodi regolarizzati (secondo regolarizzazione $L_2$) proposti dalla libreria sklearn: lbfgs, liblinear e newton-cg. La performance $R^2$ dei vari metodi in funzione del parametro di regolarizzazione, è presentata in figura \ref{fig:performance}.
\begin{figure}[H]
   \centering
   \begin{subfigure}{.35\textwidth}
      \centering
      \includegraphics[width=1 \textwidth]{immagini/istogramma_ordinato.png}
   \end{subfigure}
   \begin{subfigure}{.35\textwidth}
      \centering
      \includegraphics[width=1\textwidth]{immagini/istogramma_critico.png}
  \end{subfigure} 
  \begin{subfigure}{.35\textwidth}
   \centering
   \includegraphics[width=1\textwidth]{immagini/istogramma_disordinato.png}
   \end{subfigure} 
  \caption{\emph{Per ciascun batch, la figura mostra un istogramma della somma dei valori che compongono i sistemi. Tale somma ci fornisce una misura del livello di ordine di un sistema.}}
  \label{fig:istogrammi}
\end{figure}

\begin{figure}[H]
    \centering
    \begin{subfigure}{.35\textwidth}
       \centering
       \includegraphics[width=1 \textwidth]{immagini/lbfgs.png}
    \end{subfigure}
    \begin{subfigure}{.35\textwidth}
       \centering
       \includegraphics[width=1\textwidth]{immagini/liblinear.png}
   \end{subfigure} 
   \begin{subfigure}{.35\textwidth}
    \centering
    \includegraphics[width=1\textwidth]{immagini/newton-cg.png}
    \end{subfigure} 
   \caption{\emph{Sull'asse delle x il parametro di regolarizzazione $\lambda$, sull'asse delle y il coefficiente di determinazione $R^2$, valutato sia sul batch di test degli ordinati e disordinati, che sul batch dei critici. I tre grafici sono relativi ciascuno ad uno dei metodi di regressione.}}
   \label{fig:performance}
\end{figure}
\end{document}
%todo: perché danno tutti la stessa performance?
%todo: s