\documentclass{article}
\usepackage{parskip}
\usepackage{caption}
\usepackage{subcaption}
\usepackage{multirow}
\usepackage{float}

% Language setting
% Replace 'english' with e.g. 'spanish' to change the document language
\usepackage[italian]{babel}
\addto\captionsenglish{\renewcommand{\figurename}{Figura}}

% Set page size and margins
% Replace 'letterpaper' with 'a4paper' for UK/EU standard size
\usepackage[letterpaper,top=2cm,bottom=2cm,left=2cm,right=2cm,marginparwidth=1.75cm]{geometry}

% Useful packages
\usepackage{amsmath}
\usepackage{graphicx}
\graphicspath{ {./images/} }
\usepackage[colorlinks=true, allcolors=blue]{hyperref}


\usepackage{listings}
\usepackage{xcolor}

\definecolor{codegreen}{rgb}{0,0.6,0}
\definecolor{codegray}{rgb}{0.5,0.5,0.5}
\definecolor{codepurple}{rgb}{0.58,0,0.82}
\definecolor{backcolour}{rgb}{0.95,0.95,0.92}

\lstdefinestyle{mystyle}{
    backgroundcolor=\color{backcolour},   
    commentstyle=\color{codegreen},
    keywordstyle=\color{magenta},
    numberstyle=\tiny\color{codegray},
    stringstyle=\color{codepurple},
    basicstyle=\ttfamily\footnotesize,
    breakatwhitespace=false,         
    breaklines=true,                 
    captionpos=b,                    
    keepspaces=true,                 
    numbers=left,                    
    numbersep=5pt,                  
    showspaces=false,                
    showstringspaces=false,
    showtabs=false,                  
    tabsize=2
}

\lstset{style=mystyle}

\begin{document}
\begin{center}
    {\Large Alma Mater Studiorum - Università di Bologna}
    
    \vspace{0.5cm}
    {\bf \large Relazione per il corso di Data Science}
\end{center} 

\noindent
{ Liam Cavini} \hfill {\bf 5° Foglio, Softmax e PCA}\\
{\ Semestre Invernale 2024/2025} \hfill 19/11/2024

\subsection*{Risorse}
Il codice utilizzato, insieme al file .tex di questo documento, possono essere trovati nella seguente repository github: \url{https://github.com/LazyLagrangian/data_science}.

\subsection*{Esercizio 1 - Funzione Esponenziale Normalizzata (Softmax) e banca dati MNIST}
L'esercizio consiste nell'allenare un regressore logistico sul dataset MNIST, composto da cifre numeriche, da 0 a 9, scritte a mano su griglie di pixel $28 \times 28$.

Come nei laboratori precedenti, si è diviso il dataset in training e test, affidando alla libreria scikit-learn(sklearn) il compito di randomizzare i due batch. Sempre usando sklearn, ed in particolare il metodo di regressione logistica `sag', si è addestrato il modello e calcolata la sparsità \footnote{la percentuale dei pesi non nulli} che è mostrata in tabella \ref{tab:sparsità}.

\begin{table}[h]
    \centering
\begin{tabular}[]{|c|c|}
    \hline
    cifra & sparsità\\ \hline
    0 & 0.3648 \\ \hline
    1 & 0.2997 \\ \hline
    2 & 0.4349 \\ \hline
    3 & 0.3980 \\ \hline
    4 & 0.2755 \\ \hline
    5 & 0.3520 \\ \hline
    6 & 0.3827 \\ \hline
    7 & 0.3673 \\ \hline
    8 & 0.3546 \\ \hline
    9 & 0.2857 \\ \hline
\end{tabular}
\caption{\emph{La tabella mostra nella seconda colonna le sparsità arrotondate alla quarta cifra significativa, nella prima la cifra a cui sono associati i pesi.}}
\label{tab:sparsità}
\end{table}

Il coefficiente di determinazione in funzione dell'inverso del parametro di regolarizzazione è mostrato in figura \ref{fig:performance}.


\begin{figure}[t]
    \centering
    \includegraphics[width=0.5\textwidth]{immagini/performance.png}
    \caption{\emph{Il grafico mostra sull'asse delle ascisse l'inverso del parametro di regolarizzazione, sull'asse delle ordinate il coefficiente di determinazione $R^2$ relativo al batch di test e, separatamente, a quello di train.}}
    \label{fig:performance}
\end{figure}

I pesi relativi a ciascuna cifra sono stati visualizzati su una matrice $28 \times 28$ in figura \ref{fig:pesi}.
Si osserva che in alcuni casi i pesi più rilevanti tracciano in maniera distinguibile la cifra corrispondente.

\begin{figure}[H]
    \centering
    \begin{subfigure}[b]{0.25\textwidth}
        \centering
        \includegraphics[width=\textwidth]{immagini/coefs0.png}
    \end{subfigure}
    \begin{subfigure}[b]{0.25\textwidth}
        \centering
        \includegraphics[width=\textwidth]{immagini/coefs1.png}
    \end{subfigure}
    \begin{subfigure}[b]{0.25\textwidth}
        \centering
        \includegraphics[width=\textwidth]{immagini/coefs2.png}
    \end{subfigure}
    \begin{subfigure}[b]{0.25\textwidth}
        \centering
        \includegraphics[width=\textwidth]{immagini/coefs3.png}
    \end{subfigure}
    \begin{subfigure}[b]{0.25\textwidth}
        \centering
        \includegraphics[width=\textwidth]{immagini/coefs4.png}
    \end{subfigure}
    \begin{subfigure}[b]{0.25\textwidth}
        \centering
        \includegraphics[width=\textwidth]{immagini/coefs5.png}
    \end{subfigure}
    \begin{subfigure}[b]{0.25\textwidth}
        \centering
        \includegraphics[width=\textwidth]{immagini/coefs6.png}
    \end{subfigure}
    \begin{subfigure}[b]{0.25\textwidth}
        \centering
        \includegraphics[width=\textwidth]{immagini/coefs7.png}
    \end{subfigure}
    \begin{subfigure}[b]{0.25\textwidth}
        \centering
        \includegraphics[width=\textwidth]{immagini/coefs8.png}
    \end{subfigure}
    \begin{subfigure}[b]{0.25\textwidth}
        \centering
        \includegraphics[width=\textwidth]{immagini/coefs9.png}
    \end{subfigure}
    \caption{\emph{La figura mostra i pesi di ciascuna classe, da 0 a 9 (partendo da in alto a sinistra), su una griglia $28 \times 28$. I valori numerici corrispondenti ai colori sono consultabili dalla legenda a fianco dell'ultima immagine.}}
    \label{fig:pesi}
\end{figure}

\subsection*{Esercizio 2 - Modello di Ising e analisi delle componenti principali}
L'esercizio consiste nel compiere l'analisi dei componenti principali del modello di Ising bidimensionale discusso nel laboratorio precedente.
Prima di discutere i risultati dell'analisi, consideriamo i risultati attesi in linea teorica.

Ogni sistema può essere rappresentato come un punto in uno spazio di dimensione $n = 1600$.
Ogni coordinata può prendere il valore $1$ o $-1$, dunque i punti saranno disposti sui vertici dell'ipercubo $[-1,1]^n$.

I sistemi con temperatura bassa hanno tutti i valori uguali: di conseguenza si devono trovare sul vertice $(1,\cdots, 1)$ oppure $(-1,\cdots, -1)$.
Osserviamo che la retta passante per questi due vertici comprende anche l'origine, ed è quindi un asse, che denotiamo $\hat{n}$.
Dato che buona parte dei sistemi si trovano su questi due vertici, è ragionevole supporre che una notevole percentuale della varianza sia disposta lungo questo asse.

Ad alte temperature ciascuno dei $1600$ valori è $-1$ oppure $1$ con uguale probabilità.
I vertici su cui si collocano i punti corrispondenti a questi sistemi sono quindi molteplici, e nessun singolo asse li comprende tutti.
 

Possiamo quindi fare le seguenti predizioni:
\begin{itemize}
    \item $\hat{n}$ è un asse principale
    \item la varianza lungo $\hat{n}$ è superiore a quella lungo gli altri assi.
    \item I sistemi a basse temperature sono disposti lungo le estremità di questo asse,
    mentre quelli ad alte temperature vicino all'origine.
\end{itemize}

I risultati delle analisi confermano queste predizioni. Infatti $\hat{n}$ risulta essere il primo asse principale, ed è responsabile per circa il $50\%$ della varianza, mentre il secondo asse principale è responsabile per circa il $ 0.6\%$.
La terza predizione trova conferma in figura \ref{fig:scatter}.


\begin{figure}[htpb]
    \centering
    \includegraphics[width=0.8\textwidth]{immagini/scatter.png}
    \caption{\emph{La figura mostra la distribuzione dei sistemi lungo i primi due assi principali (con l'asse $x$ come primo), e le rispettive temperature. I valori numerici corrispondenti ai colori sono consultabili dalla legenda a fianco. Il valore di $T$ è relativo ad una arbitraria scala di energia di interazione.}}
    \label{fig:scatter}
\end{figure}

\end{document}